\documentclass[a4paper, 12pt]{article}

\usepackage[english,russian]{babel}
\usepackage[T2A]{fontenc}
\usepackage[utf8]{inputenc}

\usepackage{geometry}
\geometry{left=30mm, right=15mm, top=20mm, bottom=20mm}

\usepackage{hyperref}


\title{Лабораторная работа №5. \\
    \large Шаблоны классов С++}

\author{Автор: Никита Томашевский \thanks{При участии Павла Воронова ФКТиПМ}}
\date{\today}

\begin{document}
    \maketitle\newpage
    \tableofcontents\newpage
    \section{Постановка задачи}
    \begin{enumerate}
        \item Определить шаблон класса-контейнера:
%         \begin{enumerate}dg
%             \item Определить класс-контейнер.
%             \item Реализовать конструкторы, деструктор, операции ввода-вывода, операцию присваивания.
%             \item Перегрузить операции, указанные в варианте.
%             \item Реализовать класс-итератор. Реализовать с его помощью операции последовательного доступа.
%             \item Написать тестирующую программу, иллюстрирующую выполнение операций.
%         \end{enumerate}
        \item Реализовать конструкторы, деструктор, операции ввода-вывода, операцию присваивания.
        \item Перегрузить операции, указанные в варианте.
        \item Инстанцировать шаблон для стандартных типов данных (int, float, double).
        \item Написать тестирующую программу, иллюстрирующую выполнение операций для контейнера, содержащего элементы стандартных типов данных.
        \item Реализовать пользовательский класс (см. лабораторную работу №22).
        \begin{enumerate}
            \item Определить пользовательский класс.
            \item Определить в классе следующие конструкторы: без параметров, с параметрами, конструктор копирования.
            \item Определить в классе деструктор.
            \item Определить в классе компоненты-функции для просмотра и установки полей данных (селекторы и модификаторы).
            \item Перегрузить операцию присваивания.
            \item Перегрузить операции ввода и вывода объектов с помощью потоков.
            \item Перегрузить операции указанные в варианте.
            \item Написать программу, в которой продемонстрировать создание объектов и работу всех перегруженных операций.
        \end{enumerate}

        \item Перегрузить для пользовательского класса операции ввода-вывода.
        \item Перегрузить операции необходимые для выполнения операций контейнерного класса.
        \item Инстанцировать шаблон для пользовательского класса.
        \item Написать тестирующую программу, иллюстрирующую выполнение операций для контейнера, содержащего элементы пользовательского класса.
    \end{enumerate}
    \newpage
    \section{Постановка задачи конкретного варианта}
    \begin{enumerate}
     \item СПИСОК с ключевыми значениями типа int. Реализовать операции: []– доступа по индексу; int() – определение размера списка; + вектор – сложение элементов списков a[i]+b[i];
     \item Money для работы с денежными суммами. Число должно быть представлено двумя полями: типа long для рублей и типа int для копеек. Дробная часть числа при выводе на экран должна быть отделена от целой части запятой

    \end{enumerate}



\end{document}
